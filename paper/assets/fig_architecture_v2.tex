% Figure 1: ECG Pipeline Architecture (v2 - compact two-row layout)
% Usage: % Figure 1: ECG Pipeline Architecture (v2 - compact two-row layout)
% Usage: % Figure 1: ECG Pipeline Architecture (v2 - compact two-row layout)
% Usage: % Figure 1: ECG Pipeline Architecture (v2 - compact two-row layout)
% Usage: \input{assets/fig_architecture_v2}

\begin{figure*}[t]
\centering
\begin{tikzpicture}[
    node distance=0.5cm and 0.8cm,
    box/.style={rectangle, draw=black!70, fill=white, thick, minimum height=1.0cm, minimum width=2.2cm, align=center, font=\small},
    stage/.style={rectangle, rounded corners=4pt, draw=black!80, fill=blue!12, thick, minimum height=1.1cm, minimum width=2.6cm, align=center, font=\small\bfseries},
    arrow/.style={-{Stealth[length=3mm]}, thick, black!70},
    label/.style={font=\scriptsize\itshape, text=black!60},
    compare/.style={rectangle, rounded corners=3pt, draw=black!50, fill=gray!8, thick, minimum height=0.9cm, minimum width=2.4cm, align=center, font=\small}
]

% === ROW 1: Input → Explanation → Embed → Vectors ===
\node[box, fill=gray!15] (data) {Training Data\\$\{(x_i, y_i)\}$};

\node[stage, right=1.0cm of data] (explain) {1. LLM\\Explanation};
\node[label, above=0.1cm of explain] {Qwen3-8B};

\node[box, right=0.8cm of explain, minimum width=2.4cm] (json) {\texttt{evidence}\\[-2pt]\texttt{rationale}};

\node[stage, right=1.0cm of json] (embed) {2. Embed};
\node[label, above=0.1cm of embed] {Sentence encoder};

\node[box, right=0.8cm of embed, fill=blue!8] (vectors) {Vectors $v_i$};

% === ROW 2: kNN Graph → Surprise → Score → Output ===
% Position row 2 DIRECTLY BELOW row 1 (same x-coordinates)
\node[stage, below=1.5cm of data, xshift=3.8cm] (knn) {3. kNN Graph};
\node[label, above=0.1cm of knn] {FAISS, $k$=15};

\node[stage, right=1.0cm of knn, fill=green!18] (surprise) {4. Neighborhood\\Surprise};

\node[box, right=0.8cm of surprise, fill=green!12, minimum width=2.6cm] (score) {$\Snbr = -\log p(y_i)$};

\node[box, right=1.0cm of score, fill=red!12] (output) {Cleaned\\Dataset};

% === ARROWS: Main pipeline flow ===
% Row 1: left to right
\draw[arrow] (data) -- (explain);
\draw[arrow] (explain) -- (json);
\draw[arrow] (json) -- (embed);
\draw[arrow] (embed) -- (vectors);

% Row 1 to Row 2: vectors down to kNN (clear vertical + horizontal path)
\draw[arrow] (vectors.south) -- ++(0,-0.5) -| (knn.north);

% Row 2: left to right
\draw[arrow] (knn) -- (surprise);
\draw[arrow] (surprise) -- (score);
\draw[arrow] (score) -- (output);

% === COMPARISON PANEL: Below row 2, RIGHT-ALIGNED under output ===
\node[compare, below=1.3cm of surprise] (input_knn) {Input-kNN\\AUROC: 0.671};
\node[compare, right=0.5cm of input_knn, fill=green!18, draw=green!60!black, line width=1.2pt] (exp_knn) {\textbf{Explanation-kNN}\\AUROC: \textbf{0.832}};
\node[compare, right=0.5cm of exp_knn] (cleanlab) {Cleanlab\\AUROC: 0.107};

% Annotation below comparison
\node[label, below=0.2cm of exp_knn] {\textit{Same algorithm, different embedding space} $\rightarrow$ \textbf{+24\%}};

% === DASHED LINE: From Vectors to Explanation-kNN ===
% Route: go DOWN from vectors, then RIGHT, then DOWN to exp_knn
% This avoids overlapping with row 2 elements
\draw[dashed, black!50, thick] 
    (vectors.south) -- ++(0,-0.25) -- ++(2.0,0) |- (exp_knn.east);

\end{tikzpicture}
\caption{\textbf{ECG Pipeline.} Given training data with potentially noisy labels, ECG: (1) generates structured LLM explanations; (2) embeds the explanation text; (3) constructs a kNN graph in explanation space; (4) computes neighborhood surprise---the negative log-probability of each label given its neighbors. The key insight: the same kNN algorithm achieves \textbf{0.832 AUROC} on explanation embeddings vs.\ 0.671 on input embeddings (+24\%), while Cleanlab fails completely (0.107) on artifact-aligned noise.}
\label{fig:architecture}
\end{figure*}


\begin{figure*}[t]
\centering
\begin{tikzpicture}[
    node distance=0.5cm and 0.8cm,
    box/.style={rectangle, draw=black!70, fill=white, thick, minimum height=1.0cm, minimum width=2.2cm, align=center, font=\small},
    stage/.style={rectangle, rounded corners=4pt, draw=black!80, fill=blue!12, thick, minimum height=1.1cm, minimum width=2.6cm, align=center, font=\small\bfseries},
    arrow/.style={-{Stealth[length=3mm]}, thick, black!70},
    label/.style={font=\scriptsize\itshape, text=black!60},
    compare/.style={rectangle, rounded corners=3pt, draw=black!50, fill=gray!8, thick, minimum height=0.9cm, minimum width=2.4cm, align=center, font=\small}
]

% === ROW 1: Input → Explanation → Embed → Vectors ===
\node[box, fill=gray!15] (data) {Training Data\\$\{(x_i, y_i)\}$};

\node[stage, right=1.0cm of data] (explain) {1. LLM\\Explanation};
\node[label, above=0.1cm of explain] {Qwen3-8B};

\node[box, right=0.8cm of explain, minimum width=2.4cm] (json) {\texttt{evidence}\\[-2pt]\texttt{rationale}};

\node[stage, right=1.0cm of json] (embed) {2. Embed};
\node[label, above=0.1cm of embed] {Sentence encoder};

\node[box, right=0.8cm of embed, fill=blue!8] (vectors) {Vectors $v_i$};

% === ROW 2: kNN Graph → Surprise → Score → Output ===
% Position row 2 DIRECTLY BELOW row 1 (same x-coordinates)
\node[stage, below=1.5cm of data, xshift=3.8cm] (knn) {3. kNN Graph};
\node[label, above=0.1cm of knn] {FAISS, $k$=15};

\node[stage, right=1.0cm of knn, fill=green!18] (surprise) {4. Neighborhood\\Surprise};

\node[box, right=0.8cm of surprise, fill=green!12, minimum width=2.6cm] (score) {$\Snbr = -\log p(y_i)$};

\node[box, right=1.0cm of score, fill=red!12] (output) {Cleaned\\Dataset};

% === ARROWS: Main pipeline flow ===
% Row 1: left to right
\draw[arrow] (data) -- (explain);
\draw[arrow] (explain) -- (json);
\draw[arrow] (json) -- (embed);
\draw[arrow] (embed) -- (vectors);

% Row 1 to Row 2: vectors down to kNN (clear vertical + horizontal path)
\draw[arrow] (vectors.south) -- ++(0,-0.5) -| (knn.north);

% Row 2: left to right
\draw[arrow] (knn) -- (surprise);
\draw[arrow] (surprise) -- (score);
\draw[arrow] (score) -- (output);

% === COMPARISON PANEL: Below row 2, RIGHT-ALIGNED under output ===
\node[compare, below=1.3cm of surprise] (input_knn) {Input-kNN\\AUROC: 0.671};
\node[compare, right=0.5cm of input_knn, fill=green!18, draw=green!60!black, line width=1.2pt] (exp_knn) {\textbf{Explanation-kNN}\\AUROC: \textbf{0.832}};
\node[compare, right=0.5cm of exp_knn] (cleanlab) {Cleanlab\\AUROC: 0.107};

% Annotation below comparison
\node[label, below=0.2cm of exp_knn] {\textit{Same algorithm, different embedding space} $\rightarrow$ \textbf{+24\%}};

% === DASHED LINE: From Vectors to Explanation-kNN ===
% Route: go DOWN from vectors, then RIGHT, then DOWN to exp_knn
% This avoids overlapping with row 2 elements
\draw[dashed, black!50, thick] 
    (vectors.south) -- ++(0,-0.25) -- ++(2.0,0) |- (exp_knn.east);

\end{tikzpicture}
\caption{\textbf{ECG Pipeline.} Given training data with potentially noisy labels, ECG: (1) generates structured LLM explanations; (2) embeds the explanation text; (3) constructs a kNN graph in explanation space; (4) computes neighborhood surprise---the negative log-probability of each label given its neighbors. The key insight: the same kNN algorithm achieves \textbf{0.832 AUROC} on explanation embeddings vs.\ 0.671 on input embeddings (+24\%), while Cleanlab fails completely (0.107) on artifact-aligned noise.}
\label{fig:architecture}
\end{figure*}


\begin{figure*}[t]
\centering
\begin{tikzpicture}[
    node distance=0.5cm and 0.8cm,
    box/.style={rectangle, draw=black!70, fill=white, thick, minimum height=1.0cm, minimum width=2.2cm, align=center, font=\small},
    stage/.style={rectangle, rounded corners=4pt, draw=black!80, fill=blue!12, thick, minimum height=1.1cm, minimum width=2.6cm, align=center, font=\small\bfseries},
    arrow/.style={-{Stealth[length=3mm]}, thick, black!70},
    label/.style={font=\scriptsize\itshape, text=black!60},
    compare/.style={rectangle, rounded corners=3pt, draw=black!50, fill=gray!8, thick, minimum height=0.9cm, minimum width=2.4cm, align=center, font=\small}
]

% === ROW 1: Input → Explanation → Embed → Vectors ===
\node[box, fill=gray!15] (data) {Training Data\\$\{(x_i, y_i)\}$};

\node[stage, right=1.0cm of data] (explain) {1. LLM\\Explanation};
\node[label, above=0.1cm of explain] {Qwen3-8B};

\node[box, right=0.8cm of explain, minimum width=2.4cm] (json) {\texttt{evidence}\\[-2pt]\texttt{rationale}};

\node[stage, right=1.0cm of json] (embed) {2. Embed};
\node[label, above=0.1cm of embed] {Sentence encoder};

\node[box, right=0.8cm of embed, fill=blue!8] (vectors) {Vectors $v_i$};

% === ROW 2: kNN Graph → Surprise → Score → Output ===
% Position row 2 DIRECTLY BELOW row 1 (same x-coordinates)
\node[stage, below=1.5cm of data, xshift=3.8cm] (knn) {3. kNN Graph};
\node[label, above=0.1cm of knn] {FAISS, $k$=15};

\node[stage, right=1.0cm of knn, fill=green!18] (surprise) {4. Neighborhood\\Surprise};

\node[box, right=0.8cm of surprise, fill=green!12, minimum width=2.6cm] (score) {$\Snbr = -\log p(y_i)$};

\node[box, right=1.0cm of score, fill=red!12] (output) {Cleaned\\Dataset};

% === ARROWS: Main pipeline flow ===
% Row 1: left to right
\draw[arrow] (data) -- (explain);
\draw[arrow] (explain) -- (json);
\draw[arrow] (json) -- (embed);
\draw[arrow] (embed) -- (vectors);

% Row 1 to Row 2: vectors down to kNN (clear vertical + horizontal path)
\draw[arrow] (vectors.south) -- ++(0,-0.5) -| (knn.north);

% Row 2: left to right
\draw[arrow] (knn) -- (surprise);
\draw[arrow] (surprise) -- (score);
\draw[arrow] (score) -- (output);

% === COMPARISON PANEL: Below row 2, RIGHT-ALIGNED under output ===
\node[compare, below=1.3cm of surprise] (input_knn) {Input-kNN\\AUROC: 0.671};
\node[compare, right=0.5cm of input_knn, fill=green!18, draw=green!60!black, line width=1.2pt] (exp_knn) {\textbf{Explanation-kNN}\\AUROC: \textbf{0.832}};
\node[compare, right=0.5cm of exp_knn] (cleanlab) {Cleanlab\\AUROC: 0.107};

% Annotation below comparison
\node[label, below=0.2cm of exp_knn] {\textit{Same algorithm, different embedding space} $\rightarrow$ \textbf{+24\%}};

% === DASHED LINE: From Vectors to Explanation-kNN ===
% Route: go DOWN from vectors, then RIGHT, then DOWN to exp_knn
% This avoids overlapping with row 2 elements
\draw[dashed, black!50, thick] 
    (vectors.south) -- ++(0,-0.25) -- ++(2.0,0) |- (exp_knn.east);

\end{tikzpicture}
\caption{\textbf{ECG Pipeline.} Given training data with potentially noisy labels, ECG: (1) generates structured LLM explanations; (2) embeds the explanation text; (3) constructs a kNN graph in explanation space; (4) computes neighborhood surprise---the negative log-probability of each label given its neighbors. The key insight: the same kNN algorithm achieves \textbf{0.832 AUROC} on explanation embeddings vs.\ 0.671 on input embeddings (+24\%), while Cleanlab fails completely (0.107) on artifact-aligned noise.}
\label{fig:architecture}
\end{figure*}


\begin{figure*}[t]
\centering
\begin{tikzpicture}[
    node distance=0.5cm and 0.8cm,
    box/.style={rectangle, draw=black!70, fill=white, thick, minimum height=1.0cm, minimum width=2.2cm, align=center, font=\small},
    stage/.style={rectangle, rounded corners=4pt, draw=black!80, fill=blue!12, thick, minimum height=1.1cm, minimum width=2.6cm, align=center, font=\small\bfseries},
    arrow/.style={-{Stealth[length=3mm]}, thick, black!70},
    label/.style={font=\scriptsize\itshape, text=black!60},
    compare/.style={rectangle, rounded corners=3pt, draw=black!50, fill=gray!8, thick, minimum height=0.9cm, minimum width=2.4cm, align=center, font=\small}
]

% === ROW 1: Input → Explanation → Embed → Vectors ===
\node[box, fill=gray!15] (data) {Training Data\\$\{(x_i, y_i)\}$};

\node[stage, right=1.0cm of data] (explain) {1. LLM\\Explanation};
\node[label, above=0.1cm of explain] {Qwen3-8B};

\node[box, right=0.8cm of explain, minimum width=2.4cm] (json) {\texttt{evidence}\\[-2pt]\texttt{rationale}};

\node[stage, right=1.0cm of json] (embed) {2. Embed};
\node[label, above=0.1cm of embed] {Sentence encoder};

\node[box, right=0.8cm of embed, fill=blue!8] (vectors) {Vectors $v_i$};

% === ROW 2: kNN Graph → Surprise → Score → Output ===
% Position row 2 DIRECTLY BELOW row 1 (same x-coordinates)
\node[stage, below=1.5cm of data, xshift=3.8cm] (knn) {3. kNN Graph};
\node[label, above=0.1cm of knn] {FAISS, $k$=15};

\node[stage, right=1.0cm of knn, fill=green!18] (surprise) {4. Neighborhood\\Surprise};

\node[box, right=0.8cm of surprise, fill=green!12, minimum width=2.6cm] (score) {$\Snbr = -\log p(y_i)$};

\node[box, right=1.0cm of score, fill=red!12] (output) {Cleaned\\Dataset};

% === ARROWS: Main pipeline flow ===
% Row 1: left to right
\draw[arrow] (data) -- (explain);
\draw[arrow] (explain) -- (json);
\draw[arrow] (json) -- (embed);
\draw[arrow] (embed) -- (vectors);

% Row 1 to Row 2: vectors down to kNN (clear vertical + horizontal path)
\draw[arrow] (vectors.south) -- ++(0,-0.5) -| (knn.north);

% Row 2: left to right
\draw[arrow] (knn) -- (surprise);
\draw[arrow] (surprise) -- (score);
\draw[arrow] (score) -- (output);

% === COMPARISON PANEL: Below row 2, RIGHT-ALIGNED under output ===
\node[compare, below=1.3cm of surprise] (input_knn) {Input-kNN\\AUROC: 0.671};
\node[compare, right=0.5cm of input_knn, fill=green!18, draw=green!60!black, line width=1.2pt] (exp_knn) {\textbf{Explanation-kNN}\\AUROC: \textbf{0.832}};
\node[compare, right=0.5cm of exp_knn] (cleanlab) {Cleanlab\\AUROC: 0.107};

% Annotation below comparison
\node[label, below=0.2cm of exp_knn] {\textit{Same algorithm, different embedding space} $\rightarrow$ \textbf{+24\%}};

% === DASHED LINE: From Vectors to Explanation-kNN ===
% Route: go DOWN from vectors, then RIGHT, then DOWN to exp_knn
% This avoids overlapping with row 2 elements
\draw[dashed, black!50, thick] 
    (vectors.south) -- ++(0,-0.25) -- ++(2.0,0) |- (exp_knn.east);

\end{tikzpicture}
\caption{\textbf{ECG Pipeline.} Given training data with potentially noisy labels, ECG: (1) generates structured LLM explanations; (2) embeds the explanation text; (3) constructs a kNN graph in explanation space; (4) computes neighborhood surprise---the negative log-probability of each label given its neighbors. The key insight: the same kNN algorithm achieves \textbf{0.832 AUROC} on explanation embeddings vs.\ 0.671 on input embeddings (+24\%), while Cleanlab fails completely (0.107) on artifact-aligned noise.}
\label{fig:architecture}
\end{figure*}
